\begin{vitae}
\addcontentsline{toc}{chapter}{Curriculum Vitae}

\begin{vitaesection}{\uppercase{Education}}
\vspace{-0.1cm}
\item [2018]	Ph.D. in Environmental Science \& Managemnt (Expected), University of California, Santa Barbara.
\item [2016]	M.A. in Applied Statistics, University of California, Santa Barbara.
\item [2012] 	BS in Environmental Science and Management \textit{with honors} Emphasis in Ecology, Biodiversity and Conservation, Minor in Spanish 
University of California, Davis 
\item  [2010] Education Abroad Program Universidad de Carlos III – Madrid, Spain
\end{vitaesection}

\textbf{\uppercase{Relevant Coursework}}
\begin{tabular}{l p{0.5\linewidth}}
Probability \& Statistics (3 part course) & Regression Analysis \\
Statistical Theory (2 part course) & Statistical Consulting \\
Advanced Statistical Methodology (3 part course) & Data Mining \\
Design and Analysis of Experiments & Time Series Analysis \\
Linear and Nonlinear Mixed Effects Modeling & Data Science \\
Bayesian Data Analysis & Machine Learning (audit) \\
\end{tabular}

\textbf{\uppercase{Relevant Statistical Experience}} \\
\textbf{Quantitative Consultant,} Santa Barbara, CA \\
\textit{Bren School of Environmental Science \& Management}
\hfill
9/16 – 3/17
\vspace{-\topsep}
\begin{itemize}
\setlength{\parskip}{0pt}
 \setlength{\itemsep}{0pt plus 1pt}
\item[--] Assisted all graduate students, faculty, postdocs, and visiting researchers with their quantitative needs 
\item[--] Advised the most appropriate statistical methods to use given the data set and research questions 
\item[--] Explained how to code statistical models and interpret model results
\end{itemize}
\vspace{-\topsep}

\textbf{Probability and Statistics Department,} UCSB	 \\
\textit{Group Projects}
\hfill				         
4/16 – 6/18
\vspace{-\topsep}
\begin{itemize}\setlength{\parskip}{0pt}
\setlength{\itemsep}{0pt plus 1pt}
\setlength\itemsep{0pt plus 1pt}
\item[--] Worked on a multiple projects that utilized different data sets including: biological and political
\item[--] Select skills applied include: principal component analysis, categorical KNN, classification tree analysis (with pruning, bagging, and random forests), and Naïve Bayes 
\item[--] Lead group by setting goals and determined course of action for a group of 4
\item[--] Presented ideas effectively and wrote a report which received positive feedback from instructor
\end{itemize}
\vspace{-\topsep}

\textbf{Probability and Statistics Department,} UCSB	 \\
\textit{Individual Projects}
\hfill
12/14 – 6/16
\vspace{-\topsep}
\begin{itemize}
\setlength{\parskip}{0pt}
\setlength{\itemsep}{0pt plus 1pt}
\item[--] Worked on a variety of projects to analyze results from various data sets including: medical, economic, and biological
\item[--] Overview of skills and select a few used: fitted a time series model (Seasonal ARIMA) to data and forecasted into the future to predict values, compared the efficacy of three weight-loss programs using linear mixed effects models, looked at the economic relationship between the 48 contiguous states using multivariate analysis methods and linear models
\item[--] Presented ideas effectively in project interview and wrote a report which received positive feedback 
\end{itemize}
\vspace{-\topsep}
 
\textbf{\uppercase{STATISTICAL LEADERSHIP EXPERIENCE}} \\
\textbf{Teaching Assistant (TA),} Santa Barbara, CA \\
\textit{University of California, Santa Barbara}
\hfill
3/17 – current
\vspace{-\topsep}
\begin{itemize}
\setlength{\parskip}{0pt}
\setlength{\itemsep}{0pt plus 1pt}
\item[--] Gave multiple guest lectures to ~ 150 students on linear regression, binomial proportion test, and chi-squared tests
\item[--] Managed computer labs and showed students how to program in R, the R GUI R Commander, SAS, and Excel
\item[--] Taught material ranging from basic statistical concepts to advanced statistical theory to students who came from a wide range of backgrounds
\end{itemize}
\vspace{-\topsep}

\textbf{Statistics Tutor,} Santa Barbara, CA \\
\textit{Probability and Statistics Department}		
\hfill
9/15 – current
\vspace{-\topsep}
\begin{itemize}
\setlength{\parskip}{0pt}
\setlength{\itemsep}{0pt plus 1pt}
\item[--] Aided undergraduate students with their quantitative coursework and taught R
\item[--] Helped students understand difficult concepts by explaining it to them in novel ways
\end{itemize}
\vspace{-\topsep}

\textbf{Master’s Group Project PhD Mentor,} Santa Barbara, CA \\
\textit{Bren School of Environmental Science \& Management}
\hfill
3/16 – 3/17
\vspace{-\topsep}
\begin{itemize}
\setlength{\parskip}{0pt}
\setlength{\itemsep}{0pt plus 1pt}
\item[--] Helped master student’s set attainable goals, define research questions, and develop a feasible project timeline
\item[--] Reviewed and provided feedback on drafts of reports and presentations
\item[--] Recommended appropriate statistical analysis given data
\end{itemize}
\vspace{-\topsep}

\textbf{Intern,} La Jolla, CA \\
\textit{NOAA Southwest Fisheries Science Center}		
\hfill
8/12 – 8/13
\vspace{-\topsep}
\begin{itemize}
\setlength{\parskip}{0pt}
\setlength{\itemsep}{0pt plus 1pt}
\item[--] Analyzed scientific data and presented findings at a professional meeting
\item[--] Trained lab assistants 
\end{itemize}
\vspace{-\topsep}
 
\textbf{\uppercase{Publications}} \\
\textbf{Urbisci, L. C.}, Stohs, S. M., and Piner, K. P. 2017. From sunrise to sunset in the California drift gillnet fishery: An examination of the effects of time and area closures on the catch and catch rates of four key pelagic species: thresher shark (Alopias vulpinus), swordfish (Xiphias gladius), blue shark (Prionace glauca), and shortfin mako (Isurus oxyrinchus). Marine Fisheries Review. 78(3-4):1-12. 

Ayres, A., Degolia, A., Fienup M., Kim J., Sainz, J., \textbf{Urbisci, L. C.}, Viana, D., Wesolowski, G., Plantinga, A. J., Tague, C. 2016. Social science/natural science perspectives on wildfire and climate change. Geography Compass. 10.2: 67-86.

\textbf{Urbisci, L. C.}, Sippel, T., Teo, L. H., Piner, K. R., and Kohin, S. 2013 Size composition and spatial distribution of shortfin mako sharks by size and sex in U.S. West Coast fisheries. Submitted to ISC Shark Working Group Workshop July 6-11, 2013. 

\textbf{Urbisci, L. C.,} Runcie, R., Sippel, T., Piner, K., Dewar, H., and Kohin, S. 2012 Examining size-sex segregation among blue sharks (Prionace glauca) from the Eastern Pacific Ocean using drift gillnet fishery and satellite tagging data.  Submitted to ISC Shark Working Group Workshop January 7-14, 2013. 

\textbf{Urbisci, L. C.} 2011. Testing the unknown: the distribution, size and abundance of intertidal Haliotis rufescens (red abalone) and Haliotis cracherodii (black abalone) within Marine Protected Areas. (Unpublished student report. On file at the Cadet Hand Library, U.C. Davis Bodega Marine Laboratory). 


\textbf{\uppercase{Presentations}} \\
\textbf{Urbisci, L.C.} 2018. Untangling uncertainty in food webs. Presented to Schmidt Family Foundation on March 9, 2018, Santa Barbara, CA. 

\textbf{Urbisci, L.C.} 2017. Fishing through the food web leads to systematic overestimation of maximum sustainable yield. Presented at the NMFS-SG Annual Fellows Meeting on May 8-10, 2017, Beaufort, NC. 

\textbf{Urbisci, L.C.}, 2016. Developing an alternative estimate for virgin biomass using food web dynamics. Presented at the NMFS-SG Annual Fellows Meeting on June 28-30, 2016, Santa Cruz, CA.

\textbf{Urbisci, L.C.}, 2016. Developing an alternative estimate for virgin biomass using food web dynamics. Presented at the Bren School PhD Symposium on February 19, 2016, Santa Barbara, CA.

\textbf{Urbisci, L.C.}, 2015. Developing a new ecosystem‐based management approach: using ecosystem model to calculate a better estimate of population scale for single‐species models. Presented at the NMFS-SG Annual Fellows Meeting on June 9-11, 2015, Miami, FL.

\textbf{Urbisci, L.C.}, Stohs, S. M., and Piner, K. P. 2014. From sunrise to sunset in the California drift gillnet fishery: An examination of the effects of time and area closures on the catch and catch rates of four key pelagic species: thresher shark (Alopias vulpinus), swordfish (Xiphias gladius), blue shark (Prionace glauca), and shortfin mako (Isurus oxyrinchus). Presented at the Highly Migratory Species Management Team Meeting on January 22, 2014, La Jolla, CA.

\textbf{Urbisci, L.C.} Runcie, R., Sippel, T., Piner, K., Dewar, H., and Kohin, S. 2012 Examining size-sex segregation among blue sharks (Prionace glauca) from the Eastern Pacific Ocean using drift gillnet fishery and satellite tagging data.  Presented at the ISC Shark Working Group Workshop January 10, 2013. 

\textbf{Urbisci, L.C.} 2011. Testing the unknown: the distribution, size and abundance of intertidal Haliotis rufescens (red abalone) and Haliotis cracherodii (black abalone) within Marine Protected Areas. Presented at the Sequence One and Two Student Symposium 2011, Bodega Bay, CA.

\end{vitae}